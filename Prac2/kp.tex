\input template.tex
\initESKD{Суммирование разреженных матриц в формате RR(C)U}
\begin{document}
\newcounter{N}
\selectlanguage{russian}
\setcounter{page}{2}
\normalfont
\tableofcontents
\clearpage
\section*{ВВЕДЕНИЕ}
\addcontentsline{toc}{section}{ВВЕДЕНИЕ}
Очень часто программист работает с данными, заданными в виде матрицы (таблицы). Обработка таких данных требует довольно большого количества памяти, ведь приходится создавать двумерный массив большого размера. Однако не всегда вся исходная матрица заполнена нужными данными. Встречается такие ситуации, когда в огромной исходной таблице все элементы равны 0, за исключением малого количества. Такие матрицы называются разреженными. Хранить и обрабатывать такую таблицу в виде двумерного массива не оптимально. Желательно использовать что-то более рациональное.

В данной работе будет решаться проблема сложение разреженных матриц. Для этого будет использоваться специальный формат хранения разреженных матриц 

RR(С)U.

В данном отчёте сначала описывается сама задача, затем более подробно описывается про формат RR(C)U. 

Далее описываются структуры данных и алгоритм для написания программы.

Отчёт также содержит полный текст программы на языках C и Python, описание всех функций, инструкцию пользователю и тестовый пример. 

\clearpage
\section{ЗАДАЧА}
\subsection{Содержательное описание задачи}
Заданы две таблицы чисел, где большая часть элементов равна 0.

Требуется перевести две этих исходных матрицы в формат RR(C)U и сложить их в данном формате. Полученную  матрицу вывести в виде таблицы, а также в формате RR(C)U. 

Работа с разряженными матрицами может возникнуть в математическом анализе, а именно при решении дифференциальных уравнений в частных производных.

\subsection{Формальная постановка задачи}
Две разреженные матрицы, заданную в виде таблиц, требуется преобразовать в формат RR(C)U, сложить их и вывести на экран.

Осуществить вывод на экран как исходных, так и результирующих матриц в формате RR(C)U. Предусмотреть обнуление исходных матриц.

Рассмотрим формат представления разреженных матриц, т.е. матриц имеющих большое число нулевых элементов. В этом случае обычное представление матриц в виде массива будет избыточно, поэтому используются специальные форматы - RR(C)O и RR(C)U.  Сокращенное название первого формата происходит от английского словосочетания "Row - wise Representation Complete and Ordered" (строчное представление, полное и упорядоченное). В данном формате вместо одного двумерного массива, используются три одномерных. 
Значения ненулевых элементов матрицы и соответствующие им столбовые индексы хранятся в этом формате по строкам в двух массивах AN и JA. Массив указателей IA, используется для ссылки на компоненты массивов AN и JA, с которых начинается описание очередной строки. Последняя компонента массива IA содержит указатель первой свободной компоненты в массивах AN и JA, т.е. равна числу ненулевых элементов матрицы, увеличенному на единицу. 

Сокращенное название второго формата происходит от английского словосочетания "Row - wise Representation Complete and Unordered" (строчное представление, полное, но неупорядоченное). Формат RR(C)U отличается от RR(C)O тем, что в данном случае соблюдается упорядоченность строк, но внутри каждой строки элементы исходных матриц могут храниться в произвольном порядке. Такие неупорядоченные представления могут быть весьма удобны в практических вычислениях. Результаты большинства матричных операций получаются неупорядоченными, а их упорядочение стоило бы значительных затрат машинного времени. В то же время, за немногими исключениями, алгоритмы для разреженных матриц не требуют, чтобы их представления были упорядоченными.

В общем случае описание r-й строки матрицы A хранится в компонентах с IA[r] до IA[r + 1]-1 массивов AN и JA. Если IA[r + 1] = IA[r], то это означает, что r - я строка нулевая. Количество элементов в массиве IA на единицу больше, чем число строк исходной матрицы, а количество элементов в массивах JA и AN равно числу ненулевых элементов исходной матрицы. 

Рассмотренный формат называют полным, поскольку в нем указываются все ненулевые элементы матрицы A, упорядоченным, поскольку элементы каждой строки матрицы A хранятся по возрастанию столбовых индексов, и строчным, поскольку информация о матрице A указывается по строкам. 

Говорят, что массивы IA и JA представляют портрет (структуру) матрицы A. Если алгоритм, реализующий какую - либо операцию над разреженными матрицами, разбит на этапы символической обработки, на котором определяется портрет результирующей матрицы, и численной обработки, на котором определяются значения элементов результирующей матрицы, то массивы IA и JA заполняются на первом этапе, а массив AN - на втором. 

Рассмотрим теперь алгоритмы сложения матриц в формате RR(C)U.

На вход подаются переменные N,M - соответственно количество строк и столбцов матриц, IA,JA,AN - массивы используемые в представлении RR(C)U матрицы A. IB,JB,AB - массивы используемые в представлении RR(C)U матрицы B. На выходе получаем массивы IC,JC,CN содержащий искомую матрицу в форме RR(C)U. 

Алгоритм вначале формирует портрет матрицы С в массивах IC,JC, а затем заполняет массив CN значениями ненулевых элементов матрицы C. Можно исправить алгоритм сделав так, чтобы формирование портрета матрицы и заполнение массива CN проводилось одновременно, именно так устроен алгоритм, предъявленный ниже. 

Есть одна маленькая проблема в работе алгоритма, а именно, если для некоторых i,j выполняется a[i,j]=-b[i,j] < >0, то в представлении результирующей матрицы элемент c[i,j] должен отсутствовать, но данный алгоритм не отслеживает эту ситуацию, поэтому возможно возникновение нулевых элементов в массиве СN. 

Эта проблема решается во втором варианте алгоритма сложения RR(C)U-матриц.

В отличии от предыдущего, данный алгоритм заполняет массивы IC,JC,NC за один проход, к тому же он проверяет возникновение ситуации, когда a[i,j]=-b[i,j] < >0 и не допускает появления в массиве CN нулевых элементов, правда это скажется на скорости работы. 

Для этого алгоритм сначала проходит по строке матрицы A. Если в соответствующей строке матрицы B в массиве JB есть такой же элемент, как и в JA, то строки и столбцы этих элементов совпадают, их сумма проверяется на равенство 0, и в случае неравенства добавляется в матрицу C. Если в JB не найдено соответствующего индекса строки, то элемент из А просто добавляется в матрицу С. Добавленные индексы строк помечаются как "использованные". После того, как пройдена вся строка в матрице А, алгоритм проходит строку в матрице B; если текущий элемент JB не отмечен как "использованный", то он добавляется в матрицу С. После прохода по строке матрицы B алгоритм очищает список использованных вершин и переходит к следующей строке. 

\section{РАЗРАБОТКА АЛГОРИТМА}
\subsection{Разработка графического интерфейса пользователя}
Для ввода исходных матриц требуются две таблицы. Количество строк и столбцов должен вводить пользователь. Изменение количества строк или столбцов непосредственно в самой таблице должны вступать в силу после того, как пользователь нажмет кнопку "Новая матрица" - тогда создается новая нулевая матрица указанного размера, или кнопку "Изменить размер" - в данном случае уже введенная информация сохраняется.

Для вывода результирующей матрицы предусмотреть ещё одну таблицу. Запретить её редактирование.

Под каждой матрицей необходимо добавить три текстовых поля, для вывода матриц в формате RR(C)U (т.е. отдельные таблицы под массивы IA, JA, AN).

Создать панель меню со следующими разделами:

1)	Файл. Содержит разделы: "Выход";

2)	Правка. Содержит раздел "Создать пустые матрицы", "Изменить размер матриц".

3)	Запуск. Содержит раздел "Создать RR(C)U матрицы A","Создать RR(C)U матрицы B","Сложить RR(C)U матриц A и B".

4)  Справка. Содержит раздел "Справка".

Итак, внешний вид разработанного интерфейса представлен на рисунке \ref{INT1}.
\pic{INT1.png}{Разработанный интерфейс программы}{INT1}{H}
\subsection{Разработка структур данных}
Разреженная матрица в формате RR(C)U хранится в виде трёх массивов, следовательно, разумно создать структуру данных (RRCU) со следующими полями:

rows\_ptrs (IA) – массив типа int;

cols\_nums (JA) – массив типа int;

elements (AN) – массив типа float;

Этот тип описывает переменные, в которых хранятся две исходные матрицы, и матрица их суммы.
\subsection{Разработка структуры алгоритма}
Основную программу можно разбить на три участка: считывание значений с таблиц и сохранение их в матрицы формата RR(C)U, суммирование матриц и вывод полученной матрицы из формата RR(C)U в таблицу.

Для суммирования матриц будет создана подпрограмма sum\_rrcu, принимающая 2 параметра: матрицы типа RRCU, которые надо сложить.

\subsection{Схема алгоритма}
На рисунке \ref{LAB1} представлена схема алгоритма суммирования двух матриц в формате RR(C)U.
\pic{LAB1.png}{Схема алгоритма суммирования матриц формата RR(C)U}{LAB1}{H}
\section{РАЗРАБОТКА ПРОГРАММЫ}
\subsection{Описание переменных и структур данных}
В данной программе используются следующая структура:

RRCU – структура для хранения матрицы в формате RR(C)U со следующими полями: 

rows\_ptrs (IA) – массив типа VECTORi;

cols\_nums (JA) – массив типа VECTORi;

elements (AN) – массив типа VECTORf;

Все массивы определяются типом VECTORx, который содержит в себе:

n - int - количество элементов,

elements - массив из float(x==f) или int(x==i) - элементы массива.

Матрицы стандартного вида задаются типом MATRIX:

n,m - int - количество строк и столбцов,

elements - массив из float - элементы матрицы.

\subsection{Описание функций}
\elist{
\item RRCU sum\_rrcu(RRCU rrcu1, RRCU rrcu2)

Функция sum\_rrcu складывает матрицы rrcu1 и rrcu2, заданных в формате RR(C)U.

Возвращает матрицу - сумму исходных матриц.

Параметры  функции \ftab{sumrrcu:1}:
\tabl{Параметры  функции сложения RR(C)U}{
\tabln{rrcu1 & RRCU & первая матрица в формате RR(C)U}
\tabln{rrcu2 & RRCU & вторая матрица в формате RR(C)U}
}{sumrrcu:1}{H}

\item MATRIX expand\_rrcu(RRCU rrcu,int m)

Функция expand\_rrcu восстанавливает стандартное представление матрицы из 
матрицы rrcu, заданной в формате RR(C)U, с m столбцами.

Возвращает матрицу в стандартном представлении.
Параметры  функции \ftab{expandrrcu:1}:
\tabl{Параметры  функции восстановления из RR(C)U}{
\tabln{rrcu & RRCU & матрица в формате RR(C)U}
\tabln{m &int& количество столбцов в полном представлении матрицы.}
}{expandrrcu:1}{H}

\item RRCU create\_rrcu(MATRIX a)

Функция create\_rrcu преобразует матрицу а в формат разреженных матриц 

RR(C)U.

Возврщает матрицу в формате RR(C)U.
Параметры  функции \ftab{createrrcu:1}:
\tabl{Параметры  функции получения RR(C)U}{
\tabln{a & MATRIX & матрица в стандартном представлении.}
}{createrrcu:1}{H}
 \item input\_data(self):

Вводит исходные данные.

\item output\_data(self,rrcu\_c)

Выводит полученные данные.

\item on\_run\_click(self,button,data=None):

Производит считывание данных, отсечение и вывод результатов.
}
\section{ИНСТРУКЦИЯ ПОЛЬЗОВАТЕЛЮ}
Данная программа транспонирует заданную матрицу и выводит её на экран.

Данная программа не требует установки. Для её запуска необходимо открыть файл prac2.py. Внимание: для работы приложения на компьютере должен быть установлен Python 3, GTK+3, GObject-introspection.

Для начала требуется задать кол-во строк и столбцов в исходных матрицах. Эти числа не могут быть меньше 2. После этого нужно нажать кнопку "Создать матрицы", что приведет к созданию пустых матриц указанного размера, либо "Изменить размер", что изменит размер матриц и при этом сохранит уже введенные данные, если они входят в новые размеры. 

После ввода значений в таблицу следует выбрать пункт меню "Создать RR(C)U матрицы A","Создать RR(C)U матрицы B",или нажать кнопку под соотв. таблицей. После этого можно суммировать матрицы с помощтю меню "Сложить RR(C)U матриц A и B" или же нажать соотв. кнопку. После этого на экран будет выведена матрица суммы исходных матриц.

Внизу таблиц выводятся  массивы, описывающие эти матрицы в формате RR(C)U.

\section{ТЕСТОВАЯ ЗАДАЧА}
\subsection{Аналитическое решение и умозрительные результаты}
Возьмём матрицы 10x10. Матрица А:
$$
\begin{array}{cccccccccc}
1	&0	&0	&0	&0	&3	&0	&0	&0	&0\\
0	&0	&0	&0	&0	&0	&0	&0	&0	&0\\
0	&0	&3	&0	&0	&0	&0	&0	&0	&0\\
0	&0	&0	&0	&0	&0	&0	&0	&0	&0\\
0	&0	&0	&0	&0	&0	&0	&0	&0	&0\\
0	&0	&0	&0	&0	&6	&0	&0	&0	&0\\
0	&0	&0	&0	&0	&0	&0	&0	&0	&0\\
1	&0	&0	&0	&0	&0	&0	&0	&0	&0\\
0	&0	&0	&0	&0	&0	&0	&0	&0	&0\\
0	&0	&0	&0	&0	&0	&0	&0	&0	&0
\end{array}
$$
Матрица В:
$$
\begin{array}{cccccccccc}
1	&0	&0	&0	&0	&0	&0	&1	&0	&0\\
0	&0	&0	&0	&0	&0	&0	&0	&0	&0\\
0	&0	&3	&0	&0	&0	&0	&0	&0	&0\\
0	&0	&0	&0	&0	&0	&0	&0	&0	&0\\
0	&0	&0	&0	&0	&0	&0	&0	&0	&0\\
3	&0	&0	&0	&0	&6	&0	&0	&0	&0\\
0	&0	&0	&0	&0	&0	&0	&0	&0	&0\\
0	&0	&0	&0	&0	&0	&0	&0	&0	&0\\
0	&0	&0	&0	&0	&0	&0	&0	&0	&0\\
0	&0	&0	&0	&0	&0	&0	&0	&0	&0
\end{array}
$$

После сложения результат будет иметь вид:
$$
\begin{array}{cccccccccc}
2	&0	&0	&0	&0	&3	&0	&1	&0	&0\\
0	&0	&0	&0	&0	&0	&0	&0	&0	&0\\
0	&0	&6	&0	&0	&0	&0	&0	&0	&0\\
0	&0	&0	&0	&0	&0	&0	&0	&0	&0\\
0	&0	&0	&0	&0	&0	&0	&0	&0	&0\\
3	&0	&0	&0	&0	&12	&0	&0	&0	&0\\
0	&0	&0	&0	&0	&0	&0	&0	&0	&0\\
1	&0	&0	&0	&0	&0	&0	&0	&0	&0\\
0	&0	&0	&0	&0	&0	&0	&0	&0	&0\\
0	&0	&0	&0	&0	&0	&0	&0	&0	&0
\end{array}
$$
\subsection{Решение, полученное с использованием разработанного ПО}
Ниже на рисунке \ref{SCR1} представлен пример работы программы сложения матриц в формате RR(C)U.
\pic{SCR1.png}{Пример работы программы сложения RR(C)U-матриц}{SCR1}{H}
\subsection{Выводы}
Данная программа производит сложение матриц. Введённые значения в таблицы программа преобразует в формат RR(C)U, затем складывает матрицы в этом формате, а затем полученную матрицу из формата RR(C)U выводит в таблицу. 
\section*{ЗАКЛЮЧЕНИЕ}
\addcontentsline{toc}{section}{ЗАКЛЮЧЕНИЕ}
Разреженные матрицы требуют особого способа хранения, так как хранение их в виде массива нерационально. В написанной программе разрежённые матрицы хранятся в формате RR(C)U, в нём же и выполняются все вычисления (сложение матриц). Данный формат предоставляет экономию ресурсов и времени при работе с такими матрицами.
\section*{СПИСОК ИСПОЛЬЗОВАННЫХ ИСТОЧНИКОВ}
\addcontentsline{toc}{section}{СПИСОК ИСПОЛЬЗОВАННЫХ ИСТОЧНИКОВ}
1. http://python.org

2. http://www.gtk.org

3. http://ru.wikipedia.org

4. http://en,wikipedia.org
\section*{ПРИЛОЖЕНИЕ}
\addcontentsline{toc}{section}{ПРИЛОЖЕНИЕ}
Ниже приведен текст модуля расширения Python, реализующего работу с разреженными матрицами в формате RR(C)U и написанного на Си.
\prog{C}{alg2.c}
Далее приводится текст основной программы, написанной на Python 3.
\prog{Python}{prac2.py}
\end{document}
