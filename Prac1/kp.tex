\input template.tex
\initESKD{Отсечение отрезка прямоугольным окном}
\begin{document}
\newcounter{N}
\selectlanguage{russian}
\setcounter{page}{2}
\normalfont
\tableofcontents
\clearpage
\section*{ВВЕДЕНИЕ}
\addcontentsline{toc}{section}{ВВЕДЕНИЕ}
Компьютерная графика используется в современном мире повсеместно: реклама на ТВ, создание спецэффектов в кино, обработка фотографий, рендеринг 3d моделей.

Однако вся современная красота строго запрограммирована. Любая сложная задача требует решить сначала простую. Тривиальная задача должна быть решена с использованием минимальных ресурсов и наиболее быстро.

Одной из таких тривиальных задач в компьютерной графике считается задача отсечения отрезка прямоугольной областью. В данном отчёте описывает решение этой проблемы.

Сначала описывается сама задача, затем численный метод для её решения. Далее описываются структуры данных и алгоритм для написания программы.

Отчёт также содержит полный текст программы на языках C и Python, описание всех функций, инструкцию пользователю и тестовый пример. В написанной программе реализован алгоритм Коэна-Сазерленда.
\clearpage
\section{ЗАДАЧА ОТСЕЧЕНИЯ ОТРЕЗКА ПРЯМОУГОЛЬНЫМ ОКНОМ}
\subsection{Содержательное описание задачи}
Дана некая прямая линия, и её нужно отсечь прямоугольным окном. Данная задача может возникнуть, например, при работе с векторной графикой.

Так как вся работа с графикой на компьютере представляет собой работу с координатами, то суть задачи будет следующая: некий отрезок, заданный при помощи координат двух точек (начало отрезка и его конец) обрезать прямоугольной областью, так же заданной двумя точками (левый верхний угол и правый нижний угол). Решением данной задачи должно служить координаты двух точек обрезанной прямой.
\subsection{Формальная постановка задачи}

Задаётся отрезок, описанный координатами его концов. Так же задаётся прямоугольная область, описанная координатами двух углов: левого верхнего и правого нижнего. Требуется отсечь отрезок заданной прямоугольной областью, т.е. найти координаты начала и конца полученного отрезка.

Использовать декартовую систему координат, с координатами X и Y (т.е. плоскость).

Для решения данной задачи использовать алгоритм Коэна-Сазерленда, идея которого состоит в следующем.

Окно отсечения и прилегающие к нему части плоскости вместе образуют  9 областей. Каждой из областей присвоен 4-х битный код. 

Две конечные точки отрезка получают 4-х битные коды, соответствующие областям, в которые они попали. Смысл битов кода: 

0-ой бит = 1 –  точка левее прямоугольного окна; 

1-ый бит = 2 –  точка правее прямоугольного окна; 

2-ой бит = 4 –  точка выше прямоугольного окна; 

3-ий бит = 8 –  точка ниже прямоугольного окна. 

Определение того лежит ли отрезок целиком внутри окна или целиком вне окна выполняется следующим образом: 

 - если коды обоих концов отрезка равны 0, то отрезок целиком внутри окна, отсечение не нужно, отрезок принимается как тривиально видимый;

 - если логическое "И" кодов обоих концов отрезка не равно нулю, то отрезок целиком

 - вне окна, отсечение не нужно, отрезок отбрасывается как тривиально невидимый;
если логическое "И" кодов обоих концов отрезка равно нулю, то отрезок подозрительный, он может быть частично видимым или целиком невидимым; для него нужно определить координаты пересечений со сторонами окна и для каждой полученной части определить тривиальную видимость или невидимость. 

При расчете пересечения используется горизонтальность либо вертикальность сторон окна, что позволяет определить координату X или Y точки пересечения без вычислений. 

При непосредственном использовании описанного выше способа отбора целиком видимого или целиком невидимого отрезка после расчета пересечения потребовалось бы вычисление кода расположения точки пересечения. 

В целом схема алгоритма Сазерленда-Кохена следующая: 

1)	рассчитать коды конечных точек отсекаемого отрезка. В цикле повторять пункты 2-6;

2)	если логическое "И" кодов конечных точек не равно 0, то отрезок целиком вне окна. Он отбрасывается и отсечение закончено;

3)	если оба кода равны 0, то отрезок целиком видим. Он принимается и отсечение закончено;

4)	если начальная точка внутри окна, то она меняется местами с конечной точкой;

5)	анализируется код начальной точки для определения стороны окна с которой есть пересечение и выполняется расчет пересечения. При этом вычисленная точка пересечения заменяет начальную точку;

6)	определение нового кода начальной точки.


\section{РАЗРАБОТКА АЛГОРИТМА}
\subsection{Разработка графического интерфейса пользователя}
Для ввода координат отрезка необходимо 4 текстовых поля ввода: для задания координат начала и конца отрезка. Также для задания координат прямоугольника необходимо ещё 4 текстовых поля ввода: для задания координат двух углов: левого верхнего и правого нижнего. Вывод полученной информации осуществляется в отдельных полях-надписях.

Для наглядности нужно поместить панель, представляющей собой I квадрант декартовой системы координат для рисования исходного отрезка и прямоугольной области. Пользователь с помошью левой кнопки мыши сможет рисовать отрезок для отсечения зеленым цветом, а правой кнопкой мыши красным цветом рисуется граница области отсечения. Отсеченный отрезок рисуется красным цветом поверх исходного отрезка для большей наглядности.

В главном меню должны быть кнопки, выполняющие следующие действия: выход, выполнение отсечения, перерисовка области рисования, справка о программе.

Итак, внешний вид разработанного интерфейса представлен на рисунке \ref{INT1}.
\pic{INT1.png}{Разработанный интерфейс программы}{INT1}{H}
\subsection{Разработка структур данных}
Исходя из того, что работа осуществляется с отрезком и прямоугольным
окном, логично ввести переменные:

POINT ll\_edge,tr\_edge – левый нижний и правый верхний углы прямоугольника;
POINT b, a – точки, исходные координаты отрезка, после работы программы - координаты концов отсеченного отрезка;

Тип POINT имеет поля x и y - double - координаты точки.

Для битовых операций необходимо задать константы LEFT,TOP,BOT,RIGHT - равные различным степеням двойки.

\subsection{Разработка структуры алгоритма}
Для осуществления работы алгоритма можно выделить следующие
подпрограммы:

1) подпрограмма получения кода относительного положения точки
относительно прямоугольного окна, которой передаются углы прямоугольника и точка, код которой надо выяснить;

2) подпрограмма отсечения отрезка прямоугольным окном с помощью
алгоритма Сазерленда-Кохена, которая использует подпрограмму 3 для получения кодов положения точек отрезка. Ей передаются
начало отрезка, конец отрезка, углы прямоугольника. Программа заменяет переданные ей конец и начало исходного отрезка на координаты концов отсеченного отрезка, и возвращает 0, если он полностью лежит в прямоугольнике, 1 - если чатично, и 2 - если не лежит.

3) Подпрограмма ввода данных input\_data

4) Подпрограмма вывода данных output\_data
\subsection{Схема алгоритма}
На рисунке \ref{LAB1} представлена схема алгоритма получения кода относительного метоположения точки относительно прямоугольной области.
\pic{LAB1.png}{Схема алгоритма получения кода местополодения}{LAB1}{H}
На рисунке \ref{LAB2} представлена схема алгоритма Коэна-Сазерленда - отсечения отрезка прямоугольной областью.
\pic{LAB2.png}{Схема алгоритма Коэна-Сазерленда}{LAB2}{H}
\section{РАЗРАБОТКА ПРОГРАММЫ}
\subsection{Описание переменных и структур данных}
Исходя из того, что работа осуществляется с отрезком и прямоугольным
окном, логично ввести переменные:
POINT ll\_edge,tr\_edge – левый нижний и правый верхний углы прямоугольника;
POINT b, a – точки, исходные координаты отрезка, после работы программы - координаты концов отсеченного отрезка;

Тип POINT имеет поля x и y - double - координаты точки.

Для битовых операций необходимо задать константы LEFT,TOP,BOT,RIGHT - равные различным степеням двойки.

\subsection{Описание функций}
\elist{
\item int kohen\_sutherland(POINT *a,POINT *b,
                            POINT ll\_edge,
                            POINT tr\_edge)
							
Функция kohen\_sutherland выполняет отсечение отрезка, заданного точками a и b, 
прямоугольной областью, заданной правым верхним углом tr\_edge и 
левым нижним ll\_edge.

Возвращает 0, если исходный отрезок полностью лежит в области отсечения,
           1 - если лежит частично,
           2 - если вообще не лежит.

Функция модифицирует параметры a и b - в них возвращаются точки отсеченного отрезка.

Параметры  функции \ftab{kohen-sutherland:1}:
\tabl{Параметры  функции отсечения отрезка}{
\tabln{a,b&POINT&концы исходного отрезка}
\tabln{ll\_edge&POINT&левый нижний угол области отсечения}
\tabln{tr\_edge&POINT&правый верхний угол области отсечения}
}{kohen-sutherland:1}{H}

\item int locate(POINT p,POINT ll\_edge,POINT tr\_edge)

Функция locate определяет код точки p относительно прямоугольной области отсечения, заданной
правым верхним углом tr\_edge и левым нижним ll\_edge.

Возвращает код положения точки - TOP, RIGHT, LEFT или BOTTOM.

Параметры  функции \ftab{locate:1}:
\tabl{Параметры  функции получения битового кода}{
\tabln{p&POINT&точка для определения местоположения}
\tabln{ll\_edge&POINT&левый нижний угол области отсечения}
\tabln{tr\_edge&POINT&правый верхний угол области отсечения}
}{locate:1}{H}
\item input\_data(self):

Вводит исходные данные.

\item output\_data(self,p1,p2,ll\_edge,tr\_edge)

Выводит полученные данные.

\item on\_run\_clicked(self,button,data=None):

Производит считывание данных, отсечение и вывод результатов.

\item update\_draw(self,widget,data=None):

Обновляет рисунок отрезка и прямоугольника.

\item draw\_axes(self,cr,widget):

Рисует координатные оси.

\item draw\_clipped(self,cr,widget):

Рисует исходный и отсеченный отрезки  и прямоугольник.

}
\section{ИНСТРУКЦИЯ ПОЛЬЗОВАТЕЛЮ}
Данная программа отсекает отрезок прямоугольной областью методом Коэна-Сазерленда. Исходный отрезок задаётся двумя точками: координатами начала и конца отрезка. Исходный прямоугольник задаётся координатами двух противоположных углов. Результат выводится в виде координат отсечённого отрезка.

Данная программа не требует установки. Для её запуска необходимо открыть файл prac1.py. Внимание: для работы приложения на компьютере должен быть установлен Python версии 3, GTK+3 , GObject-introspection. Для ввода начальных данных требуется заполнить все предназначенные для этого поля цифрами, после чего нажать enter или выбрать пункт меню Запуск->Отсечь. Кроме того, можно графически нарисовать отрезок с прямоугольником. Для рисования прямоугольника на белой панели зажмите правую кнопку мыши в месте, где будет находиться левый верхний угол прямоугольника, а затем переместите курсор на позицию правого нижнего угла. Чтобы прекратить рисование, отожмите левую кнопку. Для рисования отрезка проделайте те же операции, только используете левую кнопку мыши. 

После обработки результата в нижнем текстовом окне появятся координаты начальной точки и конечной точки отсечённого отрезка. Если отрезок не входит в прямоугольную область, то об этом будет сообщено.

\section{ТЕСТОВАЯ ЗАДАЧА}
\subsection{Аналитическое решение и умозрительные результаты}
В тестовом примере возьмём прямую, заданную точками (10;10) и (70;70). Координаты точек прямоугольника следующие: (5;50) и (50;5). Очевидно, что данный отрезок обрезается правым верхним углом прямоугольника. 

Следовательно, координаты полученного отрезка следующие: (10;10) и (50;50)
\subsection{Решение, полученное с использованием разработанного ПО}
Ниже на рисунке \ref{SCR1} представлен пример работы программы отсечения отрезка прямоугольным окном.
\pic{SCR1.png}{Пример работы программы  отсечения отрезка}{SCR1}{H}
\subsection{Выводы}
Данная программа реализует алгоритм Коэна-Сазерленда. По введённым координатам отрезка и прямоугольной области производит вычисления, и в результате выдаёт координаты отсечённого отрезка.
\section*{ЗАКЛЮЧЕНИЕ}
\addcontentsline{toc}{section}{ЗАКЛЮЧЕНИЕ}
Отсечения линии прямоугольником повместно используется в компьютерной графике. Продемонстрированная программа выполняет данную задачу одним из методов, выводя результат в текстовой форме. 

Хотя данная задача и тривиальна, однако для вычисления огромного количества подобных операций требуется использовать специализированные методы, которые работают быстрее, и задействую меньше ресурсов для вычислений.

Алгоритм  Коэна-Сазерленда весьма эффективен и прост, что позволяет его использовать в мощных проектах по обработке векторной графики. 
\section*{СПИСОК ИСПОЛЬЗОВАННЫХ ИСТОЧНИКОВ}
\addcontentsline{toc}{section}{СПИСОК ИСПОЛЬЗОВАННЫХ ИСТОЧНИКОВ}
1. http://python.org

2. http://www.gtk.org

3. http://ru.wikipedia.org

4. http://en,wikipedia.org
\section*{ПРИЛОЖЕНИЕ}
\addcontentsline{toc}{section}{ПРИЛОЖЕНИЕ}
Ниже приведен текст модуля расширения Python, реализующего метод Коэна-Сазерленда, и написанного на Си.
\prog{C}{alg1.c}
Далее приводится текст основной программы, написанной на Python 3.
\prog{Python}{prac1.py}
\end{document}
