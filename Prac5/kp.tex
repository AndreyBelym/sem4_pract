\input template.tex
\initESKD{Алгоритм Йена}
\begin{document}
\newcounter{N}
\selectlanguage{russian}
\setcounter{page}{2}
\normalfont
\tableofcontents
\clearpage
\section*{ВВЕДЕНИЕ}
\addcontentsline{toc}{section}{ВВЕДЕНИЕ}
Многие структуры, представляющие практический интерес в математике и информатике, могут быть представлены графами. 

Графы часто используются для нахождения кратчайшего пути из одной точки в другую при отсутствии прямой связи между точками, учитывая возможность прохождения маршрута через какие-либо другие точки.

В данной работе разбирается алгоритм поиска некоторого числа кратчайших путей в графе - алгоритм Йена и алгоритм поиска кратчайшего пути на графе бе рёбер отрицательного веса - алгоритм Дейкстры, а также разработана программа, которая реализует этот алгоритм и визуализирует исходный граф. Отчёт содержит полный текст программы на языке Python, описание всех функций, инструкцию пользователю и тестовый пример.
\clearpage
\section{ПОИСК НЕСКОЛЬКИХ КРАТЧАЙШИХ ПУТЕЙ В ГРАФЕ}
\subsection{Содержательное описание задачи}
Дан граф без петель и без рёбер отрицательного веса. Требуется найти К кратчайших маршрутов без циклов из одной некоторой точки в другую.

Например, при К=1 ищется самый кратчайший путь. При К=2 ищется кратчайший путь, а также путь, отличающийся от кратчайшего на минимальное расстояние. Аналогично при К=3 и т.д.

Если запрошенное количество невозможно отыскать, вывести максимальное количество маршрутов (без циклов).
\subsection{Формальная постановка задачи}

Пользователь задает граф и количество кратчайших путей, которое надо найти. Требуется с помощью алгоритмов Йена и Дейкстры найти указанное количество кратчайших путей, либо сколько есть, если заданное количество слишком большое.

В математической теории графов и информатике граф — это совокупность непустого множества вершин и множества пар вершин (связей между вершинами).

Объекты представляются как вершины, или узлы графа, а связи — как дуги, или рёбра. Для разных областей применения виды графов могут различаться направленностью, ограничениями на количество связей и дополнительными данными о вершинах или рёбрах.

Для представления связей между вершинами при программировании могут использоваться различные способы.

Первый способ задания графа (невзвешенного) это задать матрицу связности S размера n*n, где n количество вершин графа, т.е. мощность множества V, при этом элемент si j=1, если существует ребро из i-ой в j-ую вершины и si j=0, если такого ребра нет. Нетрудно видеть, что матрица S- симметрична, если граф неориентированный, и может быть не симметричный в противном случае. При этом полагаем, что si i=0, т.е. в графе нет петель. 

Второй способ используется для задания взвешенного графа, т.е. графа каждому ребру которого соответствует некий параметр - вес. Для определения такого графа используется матрица весов W размер которой n*n, где n количество вершин графа. При этом элемент wi j равен весу ребра соединяющего i-ую и j-ую вершины. Если такого ребра нет, то wi j полагаем равным бесконечности (на практике это максимальное число возможное на данном языке программирования). Этот способ задания используется например в алгоритмах поиска пути во взвешенном графе. 

Алгоритм Йена предназначен для нахождения К путей минимальной длины во взвешеном графе соединяющих вершины u1,u2. Ищутся пути, которые не содержат петель. 

Задача состоит в отыскании нескольких минимальных путей, поэтому возникает вопрос о том чтобы не получить путь содержащий петлю, в случае поиска одного пути минимального веса, это условие выполняется по необходимости, в данном же случае мы используем алгоритм Йена, позволяющий находить K кратчайших простых цепей. 

Работа алгоритма начинается с нахождения кратчайшего пути, для этого будем использовать следующий алгоритм.(алгоритм Дейкстры). 

1. всем веpшинам пpиписывается вес - вещественное число, d(i)=$\inf$ для всех вершин кроме вершины с номером u1, а d(u1)=0; в качестве предыдущей вершины для всех вершин ставим начальную. 

2. всем веpшинам пpиписывается метка m(i)=0; 

3. вершина u1 объявляется текущей - t=u1 

4. для всех вершин у которых m(i)=0, пересчитываем вес по формуле: 

d(i):=min(d(i), d(t)+W[t,i]). Если вес через вершину t меньше, то запоминаем её как предыдущую для текущей вершины. 

5. среди вершин для которых выполнено m(i)=0 ищем ту для которой d(i) минимальна, если минимум не найден, т.е. вес всех не "помеченных" вершин равен бесконечности ($\inf$), то путь не существует; ВЫХОД; 

6. иначе найденную вершину c минимальным весом полагаем текущей и помечаем (m(t)=1) 

7. если t=u2, то найден путь веса d(t),ВЫХОД; 

8. переходим на шаг 4. 

После работы алгоритма мы получаем массив, в котором для каждой вершины указана предыдущая вершина кратчайшего пути. Поэтому для получения маршрута необходимо пройти по массиву от конечной вершины к начальной.

Второй путь ищем перебирая кратчайшие отклонения от первого, третий кратчайшие отклонения от второго и т.д. 

1. Найти минимальный путь P1=(v11,...,v1L[1]) .Положить k = 1. Включить P1 в результирующий список. 

2. Положить MinW равным бесконечности. Найти отклонение минимального веса, от (k–1)-го кратчайшего пути Pk-1 для всех i=1,2,...,L[k-1], выполняя для каждого i шаги с 3-го по 6-й. 

3. Проверить, совпадает ли подпуть, образованный первыми i вершинами пути Pk-1, с подпутем, образованным первыми i вершинами любого из путей j=1,2,...,k–1). Если да, положить W[vk-1i,vji+1] равным бесконечности в противном случае ничего не изменять (чтобы в дальнейшем исключить получение в результат одних и тех же путей). 

4. Используя алгоритм нахождения кратчайшего пути, найти пути от vk-1i к u2, исключая из рассмотрения корни (vk-11,...,vk-1i) (чтобы исключить петли), для этого достаточно положить равными бесконечности элементы столбцов и строк матрицы W, соответствующие вершинам входящим в корень. Можно также рассчитать вес пути до вершины i до того, как матрица будет преобразована.

5. Построить путь, объединяя корень и найденное кратчайшее ответвление, если его вес меньше MinW, то запомнить его. 6. Восстановить исходную матрицу весов W и возвратиться к шагу 3. 

7. Поместить путь минимального веса (MinW), найденый в результате выполнения шагов с 3 по 6, в результирующий список. При расчете пути можно искать его вес только от вершины i до конечной вершины, и сложить его с весом первой части пути от начала до i, полученным на шаге 4. Если k = K, то алгоритм заканчивает работу, иначе увеличить k на единицу и вернуться к шагу 2. Если MinW равен бесконечности, то больше путей невозможно найти, выход.
\section{РАЗРАБОТКА АЛГОРИТМА}
\subsection{Разработка графического интерфейса пользователя}
Для задания матрицы смежности используется квадратная таблица. Пользователь задает размер матрицы, и может создать новую пустую матрицу, либо изменяет размеры с сохранением введенных данных с помощью соотв. кнопок. Так как матрица неориентированного графа симметрична, пользователь имеет возможность отразить правый верхний треугольник матрицы с помощью соотв. кнопки.

Пользователь также вводит начальную и конечную вершины, и количество путей для поиска. 

Кнопка "Поиск" запускает алгоритм Йена для поиска кратчайших путей, которые выводятся в таблицу.

Пользователь может визуализировать граф с помощью специальной кнопки, а после нахождения путей выбрать в таблице какой-нибудь из них и также показать его на графе.

Создать панель меню со следующими разделами:

1)	Файл. Содержит разделы: "Выход";

2)	Правка. Содержит раздел "Создать пустую матрицу"\ , "Изменить размер матрицы"\ , "Сделать матрицу симметричной".

3)	Запуск. Содержит раздел "Поиск путей".

4)  Вид. Содержит "Показать граф"\ , "Показать путь".

5)  Справка. Содержит раздел "Управление graphviz-x11"\ ,"Справка".

Итак, внешний вид разработанного интерфейса представлен на рисунке \ref{INT1}.
\pic{INT1.png}{Разработанный интерфейс программы}{INT1}{H}
\subsection{Разработка структур данных}
В качестве данных будут использоватся следующие переменные:

matrix - матрица смежности заданного графа,

src,dst - начальная и конечная вершины путей,

k - количество путей для поиска.

\subsection{Разработка структуры алгоритма}
Основную программу можно разбить на три участка: считывание значений , нахождения путей и вывод полученных результатов.

1) Для нахождения нескольких кратчайших путей c помощью алгоритма Йена будет использоваться функция yen, принимающая в качестве параметров указанные в предыдущем разделе переменные, и возвращающая результат в виде списка найденных путей. Кроме того, для нахождения одного пути используется алгоритм Дейкстры, реализованной функцией dijkstra.
 
2) Подпрограмма ввода данных input\_data.

3) Подпрограмма вывода данных output\_data.
\subsection{Схема алгоритма}
На рисунке \ref{LAB1} представлена схема алгоритма Дейкстры - поиска кратчайшего пути в графе без ребер с отрицательным весом.
\pic{LAB1.png}{Схема алгоритма алгоритма Дейкстры}{LAB1}{H}
На рисунке \ref{LAB2} представлена схема алгоритма Йена - поиска K путей в графе без ребер с отрицательным весом.
\pic{LAB2.png}{Схема алгоритма алгоритма Йена}{LAB2}{H}
\section{РАЗРАБОТКА ПРОГРАММЫ}
\subsection{Описание переменных и структур данных}
В качестве данных будут использоватся следующие переменные:

matrix - [[float]] - матрица смежности заданного графа,

src,dst - int - начальная и конечная вершины путей,

k - int - количество путей для поиска.
\subsection{Описание функций}
\elist{
\item yen(matrix,src,dst,k)

Функция yen ищет не более k кратчайших путей в графе,
заданном матрицей смежности matrix,
из вершины src в вершину dst. Если в графе не может быть найдено k
кратчайших путей, возвращается лишь найденное количество.

Возвращает список длиной не более k, состоящий из кортежей,
в которых записана длина найденного пути, и
список вершин пути.
Параметры  функции \ftab{yen:1}:
\tabl{Параметры  функции поиска нескольких путей в графе}{
\tabln{matrix & [double] & матрица смежности,}
\tabln{src & int & начальная вершина,}
\tabln{dst & int & конечная вершина,}
\tabln{k & int & желаемое количество кратчайших путей.}
}{yen:1}{H}

\item dijkstra(matrix,src,dst)

Функция dijkstra ищет кратчайщий путь в графе,
заданном матрицей смежности matrix,
из вершины src в вершину dst.

Возвращает кортеж из длины найденного пути, и
списка вершин пути.
Параметры  функции \ftab{dijkstra:1}:
\tabl{Параметры  функции поиска одного пути в графе}{
\tabln{matrix & [double] & матрица смежности,}
\tabln{src & int & начальная вершина,}
\tabln{dst & int & конечная вершина}
}{dijkstra:1}{H}

\item input\_data(self)

Подпрограмма ввода исходных данных.

\item output\_data(self,paths)

Подпрограмма вывода результатов.

\item on\_run\_click(self,button,data=None)

Подпрограмма считвания данных, поиска путей и вывода результатов.

\item on\_show\_graph(self,button,data=None):

Подпрограмма построения графа.

\item on\_show\_path(self,button,data=None):

Подпрограмма построения графа и пути.
}
\section{ИНСТРУКЦИЯ ПОЛЬЗОВАТЕЛЮ}
Данная программа находит несколько кратчайших путей в графе с помощью алгоритмов Дейкстры и Йена.

Программа не требует установки. Для её запуска необходимо открыть файл prac4.py. Внимание: для работы приложения на компьютере должен быть установлен Python 3, GTK+3, GObject-introspection, Gnuplot и graphviz.

Для работы необходимо задать граф с помощью матрицы смежности. Сначала установите размер матрицы. Вы можете создавть новую пустую матрицу указанного размера, либо изменить размер с сохранением уже введенных данных. После этого можно приступать к заполнению матрицы смежности с помощью таблицы. Если исходный граф не является ориентированным, можно заполнить только верхний правый треугольник матрицы, и нажать кнопку "Сделать симметричной", так как матрица смежносте неориентированного графа является симметричной.

После заполнения матрицы смежности укажите начальную и конечную вершину путей, а также их количество. Можно также визуализировать граф с помощью graphviz, нажав специальную кнопку. После этого нажмите кнопку "Поиск".

Найденные пути выводятся в специальную таблицу. Если невозможно найти указанное количество путей, будет выведено найденное количество.

Вы можете выбрать интересующий путь в таблице и визуализировать его на графе, нажав кнопку "Показать путь".
\section{ТЕСТОВАЯ ЗАДАЧА}
\subsection{Аналитическое решение и умозрительные результаты}
Рассмотрим граф на рисунке \ref{SCR1} .
\pic{SCR1.png}{Исходный граф}{SCR1}{H}

Найдем 5 путей из вершины 1 в вершину 6.
\subsection{Решение, полученное с использованием разработанного ПО}
Ниже на рисунке \ref{SCR2} представлен пример работы программы поиска К кратчайших путей в графе.
\pic{SCR2.png}{Пример работы программы поиска кратчайших путей}{SCR2}{H}
\subsection{Выводы}
С помощью алгоритмов Дейкстры и Йена можно найти несколько кратчайших путей в графе. Программа graphviz позволяет визуализировать граф найденные пути.
\section*{ЗАКЛЮЧЕНИЕ}
\addcontentsline{toc}{section}{ЗАКЛЮЧЕНИЕ}
Графы являются наглядной и удобной струтурой данных. С помощью графа можно эффективно находить пути между двумя несвязанными напрямую точками через промежуточные.

Алгоритм Йена позволяет найти несколько кратчайших путей на графе. Для этого необходим способ поиска одного кратчайшего пути в графе, например, алгоритм Дейкстры, который ищет пути в графе без рёбер отрицательного веса.
\section*{СПИСОК ИСПОЛЬЗОВАННЫХ ИСТОЧНИКОВ}
\addcontentsline{toc}{section}{СПИСОК ИСПОЛЬЗОВАННЫХ ИСТОЧНИКОВ}
1. http://python.org

2. http://www.gtk.org

3. http://ru.wikipedia.org

4. http://en,wikipedia.org
\section*{ПРИЛОЖЕНИЕ}
\addcontentsline{toc}{section}{ПРИЛОЖЕНИЕ}
Далее приводится текст программы поиска нескольких кратчайших путей в графе, написанной на Python 3.
\prog{Python}{prac5.py}
\end{document}
