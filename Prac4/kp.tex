\input template.tex
\initESKD{Минимизация функции методом наискорейшего спуска}
\begin{document}
\newcounter{N}
\selectlanguage{russian}
\setcounter{page}{2}
\normalfont
\tableofcontents
\clearpage
\section*{ВВЕДЕНИЕ}
\addcontentsline{toc}{section}{ВВЕДЕНИЕ}
Задачей оптимизации в математике, информатике и исследовании операций называется задача нахождения экстремума (минимума или максимума) целевой функции в некоторой области конечномерного векторного пространства, ограниченной набором линейных и/или нелинейных равенств и/или неравенств. 

В данной работе разбирается оптимизация функции от двух переменных методом наискорейшего спуска, а также разработана программа, которая находит минимум функции, задаваемой пользователем в явном виде. Отчёт содержит полный текст программы на языке Python, описание всех функций, инструкцию пользователю и тестовый пример. 
\clearpage
\section{МИНИМИЗАЦИЯ ФУНКЦИИ НЕСКОЛЬКИХ ПЕРЕМЕННЫХ}
\subsection{Содержательное описание задачи}
Задана некоторая функция от двух переменных: f(x,y). Так же задано начальное приближение x0 и y0 и точность вычислений eps.
Задача состоит в том, чтобы найти минимум заданной функции (провести оптимизацию функции). Результатом должно служить минимальное значение функции, а так же соответствующие значения переменных x и y. Функция должна в явном виде вводиться пользователем.
\subsection{Формальная постановка задачи}
Задана некоторая функция от двух переменных: f(x,y). От пользователя требуется ввести следующие данные: начальное приближение x0 и y0 и точность вычислений eps. Также для надежности следует предусмотреть ограничение количества итераций алгоритма. Вводимые величины являются числами, точность вычислений eps должна быть больше 0.  Функция f(x,y) задаётся пользователем в явном виде. Требуется найти минимум функции f(x,y) с помощью метода наискорейшего спуска.

Рассмотрим алгоритм метода наискорейшего спуска Данный метод использует понятие и свойства градиента.

Градиент (от лат. gradiens, род. падеж gradientis — шагающий, растущий) — вектор, своим направлением указывающий направление наискорейшего возрастания некоторой величины , значение которой меняется от одной точки пространства к другой (скалярного поля), а по величине (модулю) равный быстроте роста этой величины в этом направлении.

Например, если взять в качестве  высоту поверхности Земли над уровнем моря, то её градиент в каждой точке поверхности будет показывать «направление самого крутого подъёма», и своей величиной характеризовать крутизну склона.

С математической точки зрения градиент — это производная скалярной функции, определенной на векторном пространстве.

Пространство, на котором определена функция и её градиент может быть вообще говоря как обычным трехмерным пространством, так и пространством любой другой разменрости любой физической природы или чисто абстрактным.

Термин впервые появился в метеорологии, а в математику был введен Максвеллом в 1873 г. Обозначение $grad$ тоже предложил Максвелл.

Т.е. если градиент показывает направление наискорейшего роста функции, то антиградиент - градиент с противоположным знаком - показывает направление наискорейшего убывания функции (в данной точке). Это свойство антиградиента лежит в основе градиентных методов, в частности, метода наискорейшего спуска.

Для нахождения минимума $F$ задаем некоторое начальное приближение $x_i^{(0)} (i=1,...,n)$ и строим последующие приближения по формуле: 
	
$$x_{i}^{(j+1)}=x_{i}^{(j)}+\lambda_{i}^{(j)}v_{i}^{(j)} (i=1,...,n;j=0,1,2..)$$

где направления $v_i^{(j)}$ и величина шага на j-м шаге соответственно равны: 
$$v_i^{(j)}=-\frac{\partial F} {\partial x_i}$$
$$\lambda_i^{(j)}=\sum_i \left( \frac{\partial F} {\partial x_i}\right)^2 \left[\sum_{i,j} \frac{\partial^2 F} {\partial x_i \partial x_j} \frac{\partial F} {\partial x_i} \frac{\partial F} {\partial x_j} \right]^{-1} $$	
Все производные вычисляются при $x_i=x_i^{(j)}$. 

Итерационный процесс продолжается до тех пор, пока не будет удовлетворяться условие 
	$$|x_i^{(j+1)}-x_i^{(j)}| \le e; (i=1,...,n)$$
или все производные $\frac{\partial F}{\partial x_k}$ не станут равны нулю.  
\section{РАЗРАБОТКА АЛГОРИТМА}
\subsection{Разработка графического интерфейса пользователя}
Для решения задачи требуются иметь следующие исходные данные: начальное приближение переменных x и y, а также точность вычислений eps и максимальное кол-во итераций. Для ввода этих значений необходимо предусмотреть отдельные поля. Для ввода функции f(x,y) также предусмотреть поле, причем необходимо обеспечить ввод в явном виде: например $f(x,y)=5x+4y^2+2$. Кроме того, можно предоставить пользователю возможность скомпилировать исходную функцию в код на Фортране, что позволит сильно увеличить скорость вычислений. Для переключения между режимами интерпретации и компиляции функции предусмотреть элемент ComboBox. Так как компиляция занимает довольно продолжительное время, предусмотреть отдельную кнопку для принятия функции, и продублировать её в меню "Запуск".

Для того, чтобы можно было проследить динамику работы метода, приближения, полученные на каждом шаге метода, будут выводится в таблицу. Кроме того, будет построен график исходной функции с помощью программы Gnuplot; на этом графике так же соединёнными точками будут показываться приближения, полученные с помощью метода.

Для вычисления результата создать кнопку "Расчет" и продублировать её в пункте меню Запуск->Найти минимум. В панели меню предусмотреть пункт для открытия окна “Справка” и справки по управлению программой Gnuplot, "Вид"->"Показать график" и пункт меню Файл->Выход. 

Итак, внешний вид разработанного интерфейса представлен на рисунке \ref{INT1}.
\pic{INT1.png}{Разработанный интерфейс программы}{INT1}{H}
\subsection{Разработка структур данных}
Для хранения исходных данных будем использовать следующие переменные:

max\_iter - максимальное количество итераций,

eps - точность вычислений,

xs - начальные приближения,

f - минимизируемая функция,

df\_dx - первые производные минимизируемой функции,

d2f\_dx - матрица вторых производных (матрица Гессе) функции.
\subsection{Разработка структуры алгоритма}
Основную программу можно разбить на три участка: считывание значений , нахождения минимума функции и вывод полученных результатов.

1) Для нахождения минимума функции будет использоваться функция gradient, принимающая в качестве параметров указанные в предыдущем разделе переменные, и возвращающая результат в виде списка приближений к минимуму. Приближения представляют собой кортежи длиной n для функции n переменных.

2) Подпрограмма ввода данных input\_data.

3) Подпрограмма вывода данных output\_data.

\subsection{Схема алгоритма}
На рисунке \ref{LAB1} представлена схема алгоритма минимизации функции многих переменных градиентным методом наискорейшего спуска.
\pic{LAB1.png}{Схема алгоритма метода наискорейшего спуска}{LAB1}{H}
\section{РАЗРАБОТКА ПРОГРАММЫ}
\subsection{Описание переменных и структур данных}
Для хранения исходных данных будем использовать следующие переменные:

max\_iter - float - максимальное количество итераций,

eps - float - точность вычислений,

xs - (float) - начальные приближения,

f - UserFunc - минимизируемая функция,

df\_dx - [UserFunc] - первые производные минимизируемой функции,

d2f\_dx - [[UserFunc]] - матрица вторых производных (матрица Гессе) функции.

Класс UserFunc представляет функцию, задаваемую с помощью строки.

\subsection{Описание функций}
\elist{
\item gradient(max\_iter,eps,xs,f,df\_dx,d2f\_dx)

Функция gradient проводит минимизацию функции f,
используя начальное приближение xs, первые производные df\_dx,
вторые производные d2f\_dx, при максимальном кол-ве итераций max\_iter
и точностью eps.

Возвращает список, содержащий приближения к минимуму функции.

Параметры  функции \ftab{gradient:1}:
\tabl{Параметры  функции минимизации}{
\tabln{max\_iter & int & максимальное кол-во итераций,}
\tabln{eps & float & максимальная разница между соседними приближениями,}
\tabln{xs & (double) & кортеж начальных приближений,}
\tabln{f & function & минимизируемая функция,}
\tabln{df\_dx & [function] & список первых производных функции,}
\tabln{d2f\_dx & [[function]] & матрица вторых производных (матрица Гессе)}
}{gradient:1}{H}

\item input\_data(self)

Подпрограмма ввода исходных данных.

\item output\_data(self,res)

Подпрограмма вывода результатов.

\item on\_run\_click(self,button,data=None)

Подпрограмма считвания данных, минимизации функции и вывода результатов.

\item show\_chart(self):

Подпрограмма построения графика.

}
\section{ИНСТРУКЦИЯ ПОЛЬЗОВАТЕЛЮ}
Данная программа находит минимум функции f(x,y).
Программа не требует установки. Для её запуска необходимо открыть файл prac4.py. Внимание: для работы приложения на компьютере должен быть установлен Python 3, GTK+3, GObject-introspection, Gnuplot и SymPy.

От пользователя требуется ввести следующие исходные данные: 

1) Функция f(x,y). После её ввода необходимо выбрать режим вычисления (интерпретация или компиляция) и нажать кнопку "Принять".

2) Начальное приближение x0.

3) Начальное приближение y0.

4) Точность вычислений eps.

5) Максимальное количество итераций max\_iter.

Примечание: Функция вводится в явном виде, т.е. в поле f(x,y) можно ввести "(x-1)**2+100*(y-x**2)**2". Учтите, точность вычислений доложена быть больше 0.
После ввода значений для получения результата требуется либо открыть пункт меню  Запуск->Минимизировать функцию, либо нажать на кнопку «Расчет!». После этого на экран будут выведен найденные приближения к минимуму функции. Учтите, нахождение минимума функции может занять несколько минут.

\section{ТЕСТОВАЯ ЗАДАЧА}
\subsection{Аналитическое решение и умозрительные результаты}
Введём функцию Розенброка: $$f(x,y)=(x-1)^2+100(y-x^2)^2$$, стандартно используемую для тестирования методов оптимизации.

Её минимум находится в точке $(1,1)$ и равен 0.
\subsection{Решение, полученное с использованием разработанного ПО}
Ниже на рисунке \ref{SCR1} представлен пример работы программы  минимимизации функции двух переменных методом наискорейшего спуска.
\pic{SCR1.png}{Пример работы программы минимизации функции}{SCR1}{H}
\subsection{Выводы}
Данная программа находит минимум функции от двух переменных методом наискорейшего спуска. 

Так как функция задается строкой, а используемый метод требует задания производных первого и второго порядка, для их расчета используется библиотека символьной алгебры SymPy.
\section*{ЗАКЛЮЧЕНИЕ}
\addcontentsline{toc}{section}{ЗАКЛЮЧЕНИЕ}
В данной работе рассматривалась проблема оптимизации функции двух переменных. Для решения проблемы был использован метод наискорейшего спуска. Была написана программа на языке Python, реализующий данный метод. Примечательно то, что исходная функция в явном виде вводится пользователем, что делает программу намного универсальнее.
\section*{СПИСОК ИСПОЛЬЗОВАННЫХ ИСТОЧНИКОВ}
\addcontentsline{toc}{section}{СПИСОК ИСПОЛЬЗОВАННЫХ ИСТОЧНИКОВ}
1. http://python.org

2. http://www.gtk.org

3. http://ru.wikipedia.org

4. http://en,wikipedia.org
\section*{ПРИЛОЖЕНИЕ}
\addcontentsline{toc}{section}{ПРИЛОЖЕНИЕ}
Далее приводится текст программы оптимизации функции двух переменных методом наискорейшего спуска, написанной на Python 3.
\prog{Python}{prac4.py}
\end{document}
