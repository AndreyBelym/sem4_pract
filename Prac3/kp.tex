\input template.tex
\initESKD{Интегральные уравнения Фредгольма второго рода}
\begin{document}
\newcounter{N}
\selectlanguage{russian}
\setcounter{page}{2}
\normalfont
\tableofcontents
\clearpage
\section*{ВВЕДЕНИЕ}
\addcontentsline{toc}{section}{ВВЕДЕНИЕ}
Интегральными уравнениями называются функциональные уравнения, содержащие интегральные преобразования над неизвестной функцией y(x). 

Известны несколько видов интегральных уравнений (уравнение Вольтера, уравнение Фредгольма). Решение их может возникнуть в математическом анализе.

В данной работе разбирается неоднородное интегральное уравнение Фредгольма второго рода. Рассматривается вид этого уравнения и способ его решения.  

Отчёт также содержит полный текст программы на языках C и Python, описание всех функций, инструкцию пользователю и тестовый пример. 

\clearpage
\section{РЕШЕНИЕ ИНТЕГРАЛЬНЫХ УРАВНЕНИЙ ФРЕДГОЛЬМА ВТОРОГО РОДА}
\subsection{Содержательное описание задачи}
Неоднородное уравнение Фредгольма второго рода выглядит так:

$$f(x)=\phi(x)-\lambda \int_a^b K(x,s)\phi(s)ds$$

Задача состоит в том, чтобы имея ядро $K(x,s)$ и функцию $f(x)$, найти функцию $\phi(x)$. То есть, задана некая функция $f(x)$, задано ядро, представляющее собой функцию от двух переменных $K(x,s)$. Также заданы пределы интегрирования $a$ и $b$, коэффициент $\lambda$ и шаг интегрирования. Требуется найти табличные значения функции $\phi(x)$, где соответствующим значениям $x$ изменяющимся от $a$ до $b$ с заданным шагом сопоставлены соответствующие значения функции.

\subsection{Формальная постановка задачи}
Неоднородное уравнение Фредгольма второго рода выглядит так:

$$f(x)=\phi(x)-\lambda \int_a^b K(x,s)\phi(s)ds$$

Пусть функции $f(x)$ и $K(x,s)$ задаются в явном виде. Значения переменных $a$,$b$, $\lambda$ и шаг интегрирования должны быть числами, причём шаг интегрирования должен быть больше 0, $a<=b$.

Ответ представить в виде таблице с тремя столбцами: в первом находится аргумент искомой функции изменяющимся от $a$ до $b$ с заданным шагом, а во втором столбце соответствующие аргументу значения функции, в третьем столбце представленно значение аналитического решения. Для наглядности должен быть представлен график искомой функции.

Для решения задачи требуется знать описанные ниже теоретические сведения. Интегральными уравнениями называются функциональные уравнения, содержащие интегральные преобразования над неизвестной функцией $y(x)$. Интегральное уравнение называется однородным, если $ay(x)$ есть решение уравнения для произвольного $a$. Линейное интегральное уравнение в общем виде может быть представлено: 

$$g(x)y(x)-a\int_v k(x,s)y(s)ds=f(x)$$
	 
где $k(x,s)$ - ядро интегрального преобразования, правая часть $f(x)$ и $g(x)$ являются заданными функциями, $a$ - параметр уравнения. Область интегрирования $V$ может быть фиксированной (интегральные уравнения типа фредгольмовых) или переменной (интегральные уравнения типа вольтерровых). 
Линейное интегральное уравнение первого рода получается при $g(x)=0$, $a=-1$ и имеет вид: 
	 
     $$\int_v k(x,s)y(s)ds=f(x)$$

Однородное линейное интегральное уравнение второго рода получается при $f(x)=0$,$g(x)=1$ и имеет вид: 
	 
$$y(x)-\int_v k(x,s)y(s)ds=0$$

Неоднородное интегральное уравнение второго рода получается при $g(x)=1$ и имеет вид: 
	 
$$y(x)-\int_v k(x,s)y(s)ds=f(x)$$

Уравнения вида 
	 
$$y(x)-\int_v k(x,s)F(y(s))ds=f(x)$$


являются неоднородными. 

Линейное интегральное неоднородное уравнение Фредгольма второго рода имеет вид: 
	 
$$y(x)-\lambda\int_a^b k(x,s)F(y(s))ds=f(x)$$

где ядро определено в квадрате $V=[a,b]*[a,b]$. Кроме того, полагается, что ядро непрерывно в $V$. При  $\lambda=1$, используя квадратурную формулу трапеций с постоянным шагом h, получим: 
	 
$$y_i-h\sum_{j=1}^n A_j k_{ij} y_j = f_i$$

где $n=(b-a)/h+1$-целое, $A_j=1$ при $j$ не равном 1 или $n$ и $A_j=0.5$ при $j=1$ или $n$. 

\section{РАЗРАБОТКА АЛГОРИТМА}
\subsection{Разработка графического интерфейса пользователя}
Для решения задачи требуются иметь следующие исходные данные: начало интегрирования (a), конец интегрирования (b), шаг интегрирования. Для ввода этих значений необходимо предусмотреть отдельные поля. Функции f(x) и K(x,s) будут заданы программно, поэтому нужно обеспечить вывод этих функций на экран. Кроме того, для наглядности и проверки правильности работы программы будет использоваться уравнение с уже известным аналитическим решением, и эта функция-решение также должна быть выведена на экран.

Известно, что результатом вычислений должна быть функция, следовательно, необходима таблица для вывода аргументов и значений полученной функции, и значений аналитического решения. Также решения (дискретное и аналитическое) будут визуализироваться с помощью программы Gnuplot, которая запускается в отдельном окне по нажатию специальной кнопки.

В панели меню нужно предусмотреть следующие пункты: выход из программы, решение уравнения, вывод графиков, справка по программе Gnuplot и справка по данной программе. 

Итак, внешний вид разработанного интерфейса представлен на рисунке \ref{INT1}.
\pic{INT1.png}{Разработанный интерфейс программы}{INT1}{H}
\subsection{Разработка структур данных}
Для описания исходного уравнения Фредгольма будем использовать следующие переменные

a – Начало интегрирования.

b – Конец интегрирования. 

h – шаг для интегрирования. 

F(x) – табличное представление функции f(x)

K(x,s) – табличное представления ядра K(x,s)
\subsection{Разработка структуры алгоритма}
Основную программу можно разбить на три участка: считывание значений, нахождения табличных значений искомой функции и вывод полученных результатов.

1) Для нахождения табличных значений искомой функции создадим функцию fredholm, параметры решения уравнения (см. предыдущий разел). Возвращать данная функция будет массив, представляющий собой значения искомой функции. Так как для решения уравнения Фредгольма требуется решить систему линейных алгебраических уравнений, следует создать отдельную подпрограмму для нахождения корней СЛАУ. В данной работе будет использоваться метод Гаусса для решения СЛАУ. 

2) Подпрограмма ввода данных input\_data.

3) Подпрограмма вывода данных output\_data.
\subsection{Схема алгоритма}
На рисунке \ref{LAB2} представлена схема алгоритма создания системы алгебраических уравнений для решения интегрального уравнения Фредгольма.
\pic{LAB2.png}{Схема алгоритма создания системы уравнений}{LAB2}{H}
На рисунке \ref{LAB1} представлена схема алгоритма решения интегрального уравнения Фредгольма.
\pic{LAB1.png}{Схема алгоритма решения уравнения Фредгольма}{LAB1}{H}
\section{РАЗРАБОТКА ПРОГРАММЫ}
\subsection{Описание переменных и структур данных}
В данной программе используются следующие переменные:

a – double - Начало интегрирования.

b – double - Конец интегрирования. 

h – double - шаг для интегрирования. 

F(x) – VECTOR - табличное представление функции f(x)

K(x,s) – MATRIX - табличное представления ядра K(x,s)

Все массивы определяются типом VECTOR, который содержит в себе:

n - int - количество элементов,

elements - массив из float - элементы массива.

Матрицы стандартного вида задаются типом MATRIX:

n,m - int - количество строк и столбцов,

elements - массив из float - элементы матрицы.

\subsection{Описание функций}
\elist{
\item MATRIX create\_sys(MATRIX k,VECTOR f,double h,double a,double b)

Функция create\_sys создает матрицу системы для решения уравнения Фредгольма
второго рода,  используя таблично-заданные 
ядро k\_d и свободную составляющую f\_d, с шагом h на интервале [a,b].

Возвращает матрицу системы.

Параметры  функции \ftab{createsys:1}:
\tabl{Параметры  функции создания матрицы системы}{
\tabln{k & MATRIX & таблично-заданное ядро уравнения}
\tabln{f & VECTOR & таблично-заданная свободная функция}
\tabln{h & double & шаг поиска решения}
\tabln{a,b & double & интервал поиска решения}
}{createsys:1}{H}

\item VECTOR solve\_eq(MATRIX k,VECTOR f,double h, double a,double b)

Функция solve\_eq решает интегральное уравнение Фредгольма второго рода, используя таблично-заданные 
ядро k\_d и свободную составляющую f\_d, с шагом h на интервале [a,b].

Возвращает вектор решения уравнения.

Параметры  функции \ftab{solveeq:1}:
\tabl{Параметры  функции решения уравнения}{
\tabln{k & MATRIX & таблично-заданное ядро уравнения}
\tabln{f & VECTOR & таблично-заданная свободная функция}
\tabln{h & double & шаг поиска решения}
\tabln{a,b & double & интервал поиска решения}
}{solveeq:1}{H}

\item VECTOR gauss(MATRIX matr)

Функция решения СЛАУ методом Гаусса.

\item input\_data(self)

Подпрограмма ввода исходных данных.

\item output\_data(self,a,b,x,y)

Подпрограмма вывода результатов.

\item on\_run\_click(self,button,data=None)

Подпрограмма считвания данных, решения уравнения и вывода результатов.

\item show\_chart(self):

Подпрограмма построения графика.
}
\section{ИНСТРУКЦИЯ ПОЛЬЗОВАТЕЛЮ}
Данная программа решает интегральное уравнение Фредгольма второго рода вида: 
$$
y(x)-{\frac{1}{2}}\int_a^b xe^s y(s) ds = e^{-x}
$$ .

Данная программа не требует установки. Для её запуска необходимо открыть файл prac3.py. Внимание: для работы приложения на компьютере должен быть установлен Python 3, GTK+3, GObject-introspection  и Gnuplot.

Для работы необходимы следующие данные:

.	Начало промежутка интегрирования 

	Конец промежутка интегрирования 

	Шаг интегрирования 

После ввода значений для получения результата требуется нажать кнопку "Решить" либо открыть пункт меню  Запуск->Решить уравнение. После этого на экран будут выведены табличные значения искомой функции и построен соответствующий график с помощью программы Gnuplot.

\section{ТЕСТОВАЯ ЗАДАЧА}
\subsection{Аналитическое решение и умозрительные результаты}
Данная программа решает интегральное уравнение Фредгольма второго рода вида: 
$$
y(x)-{\frac{1}{2}}\int_a^b xe^s y(s) ds = e^{-x}
$$ .

Пусть дан участок интегрирования от 0 до 1.

Тогда аналитическое решение уравнение:
$$
y(x)=e^{-x}+x
$$

\subsection{Решение, полученное с использованием разработанного ПО}
Ниже на рисунке \ref{SCR1} представлен пример работы программы решения интегрального уравнения Фредгольма второго рода.
\pic{SCR1.png}{Пример работы программы решения уравнения Фредгольма}{SCR1}{H}
\subsection{Выводы}
Данная программа решает интегральное уравнение Фредгольма второго рода. Ввод функции f(x) и K(x,s)  осуществляется в явном виде. Так как решением уравнения является функция, то программы выводит  табличные значения искомой функции и строит по ним график.
\section*{ЗАКЛЮЧЕНИЕ}
\addcontentsline{toc}{section}{ЗАКЛЮЧЕНИЕ}
Решение интегральных уравнений, в частности уравнений Фредгольма второго рода, представляет собой довольно сложную алгебраическую задачу. С помощью рассмотренного дискретного метода можно получить решение данного типа уравнений с любой требуемой точностью.
\section*{СПИСОК ИСПОЛЬЗОВАННЫХ ИСТОЧНИКОВ}
\addcontentsline{toc}{section}{СПИСОК ИСПОЛЬЗОВАННЫХ ИСТОЧНИКОВ}
1. http://python.org

2. http://www.gtk.org

3. http://ru.wikipedia.org

4. http://en,wikipedia.org
\section*{ПРИЛОЖЕНИЕ}
\addcontentsline{toc}{section}{ПРИЛОЖЕНИЕ}
Ниже приведен текст модуля расширения Python, реализующего решение интегрального уравнения Фредгольма второго рода и написанного на Си.
\prog{C}{alg3.c}
Далее приводится текст основной программы, написанной на Python 3.
\prog{Python}{prac3.py}
\end{document}
