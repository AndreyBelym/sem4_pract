\documentclass[utf8,nocolumnxxxi,nocolumnxxxii]{eskdtext}
\usepackage{eskdtotal}
\renewcommand{\tiny}{\fontsize{12}{14}\fontfamily{ftm}\linespread{1.0}\selectfont}
\newcommand{\initESKD}[1]{
\ESKDcolumnI{\tiny Пояснительная записка к 
лабораторной работе по курсу 
«Вычислительный практикум» по теме 
«#1»}
}
\ESKDcolumnII{Вариант №3}
\ESKDcolumnIX{ТулГУ гр. 220601}
\ESKDcolumnXIfI{Белым А.А.}
\ESKDcolumnXIfII{Ермаков А.С.}
\pagestyle{empty}
\usepackage{layout}
\usepackage{amsmath}
%\usepackage[left=2.5cm, right=1cm, bottom=1cm, top=1cm]{geometry}
\parindent=1.0cm
\footskip=0.2cm
%\parskip=-0.8cm
\usepackage[utf8]{inputenc}
\usepackage[T2A]{fontenc}
\usepackage[russian]{babel}
%\usepackage{pscyr}
%\usepackage[compact,center,small]{titlesec}
%\titlelabel{}
%\titlespacing{\section}{0.0cm}{-0.3cm}{-0.5cm}
\usepackage{listingsutf8}
\usepackage{graphicx}
%\usepackage{ccaption}
\usepackage{tabularx}
\usepackage{array}
%\captiondelim{ - }
\renewcommand{\baselinestretch}{1.25}
\renewcommand{\normalfont}{\fontsize{14}{20}\fontfamily{ftm}\linespread{1.25}\selectfont}
\renewcommand{\thesection}{\arabic{section}.} 
\renewcommand{\thesubsection}{\arabic{section}.\arabic{subsection}.}
\ESKDsectStyle{section}{\normalfont\fontfamily{far}\selectfont\bfseries\centering}
\ESKDsectStyle{subsection}{\normalfont\fontfamily{ftm}\selectfont\bfseries\centering}
\ESKDsectSkip{subsection}{1.5cm}{0.5cm}
\newcommand{\maketitlexx}[2]{
\begin{titlepage}
\begin{center}\linespread{1}\parskip=0.0cm\normalfont
\ESKDthisStyle{empty}
Министерство образования и науки РФ

ФГБОУ ВПО Тульский государственный университитет

\vskip 0pt plus 0.5fil
КАФЕДРА АВТОМАТИКИ И ТЕЛЕМЕХАНИКИ

\vfill
\textbf{#1}

\vskip 2cm
Пояснительная записка 

к лабораторной работе № #2


по курсу «Вычислительный практикум»

\vfill
Вариант № 3

\vfill
\begin{tabular*}{\textwidth}{ll@{\extracolsep{\fill}}c@{\extracolsep{0pt}}l}
Выполнил: & студент группы 220601&&Белым~А.А.\\ \cline{3-3}
								&&\tiny{(подпись)}& 	\\
Проверил: & к. ф.-м. н., доцент &&Ермаков А.С.\\ \cline{3-3}
								&&\tiny{(подпись)}& 	\\
\end{tabular*}
\vfill
Тула 2012
\end{center}
\end{titlepage}
}
\usepackage{float}
\newcommand{\pic}[4]{
\renewcommand{\figurename}{\tinyРисунок}
%\suppressfloats[p]
\begin{figure}[#4]
\center
\includegraphics{#1}
\caption{\tiny #2}\label{#3}
\end{figure}
}
\newcommand{\tabln}[1]{#1 \\ \hline}
\newcommand{\tabl}[4]{
\renewcommand{\tablename}{\hfill \tiny Таблица}
\begin{table}[#4]
\caption{\tiny #1}\label{#3}
\center\normalfont\linespread{1.0}\selectfont
\vskip -0.5cm 
\begin{tabular}{|c|c|l|} \hline
{\bfseries имя}&{\bfseries тип} &\multicolumn{1}{c|}{\bfseries предназначение}  \\ \hline
#2
\end{tabular}
\end{table}
}
\newcommand{\ftab}[1]{
 представлены в таблице \ref{#1}
}
\newcommand{\tabbox}[2]{\parbox[t]{#1}{\linespread{0.8}\selectfont #2}}
\newcommand{\prog}[2]{
\fontencoding{T2A}

\lstset{basicstyle=\fontsize{10}{12}\fontfamily{fcr}\linespread{1.0}\selectfont,identifierstyle=,keywordstyle=\bfseries,commentstyle=\itshape,stringstyle=\slshape}
\lstinputlisting[inputencoding=utf8/cp1251,language=#1]{#2}
}
\newcommand{\elist}[1]{
\suppressfloats[t]
\begin{list}{\arabic{N}.}{\itemindent 0.8cm\addtolength{\itemindent}{\labelwidth}\listparindent 1.0cm\leftmargin 0cm\usecounter{N}}

#1
\end{list}
\setcounter{N}{0}}

\makeatletter
%\renewcommand{\@oddfoot}{{\quad\hfill\quad \thepage}}
%\renewcommand{\@evenfoot}{{\quad\hfill\quad \thepage}}
\makeatother
\newcommand{\ssec}[1]{\section{#1}\hspace*{\parindent}}
\usepackage{tocloft}
\tocloftpagestyle{empty}
\renewcommand{\cftsecleader}{\cftdotfill{\cftdotsep}}
\renewcommand{\cfttoctitlefont}{\hfill\fontfamily{far}\selectfont\bfseries}
\renewcommand{\cftaftertoctitle}{\hfill}
