\input template.tex
\initESKD{Отсечение отрезка прямоугольным окном}
\begin{document}
\selectlanguage{russian}
%\setcounter{page}{2}
\normalfont
\tableofcontents
\clearpage
\section*{ВВЕДЕНИЕ}
\addcontentsline{toc}{section}{ВВЕДЕНИЕ}
\clearpage
\section{ЗАДАЧА}
\subsection{Содержательное описание задачи}
\subsection{Формальная постановка задачи}
\section{РАЗРАБОТКА АЛГОРИТМА}
\subsection{Разработка графического интерфейса пользователя}
\subsection{Разработка структур данных}
\subsection{Разработка структуры алгоритма}
\subsection{Схема алгоритма}
\section{РАЗРАБОТКА ПРОГРАММЫ}
\subsection{Описание переменных и структур данных}
\subsection{Описание функций}
\section{ИНСТРУКЦИЯ ПОЛЬЗОВАТЕЛЮ}
\section{ТЕСТОВАЯ ЗАДАЧА}
\subsection{Аналитическое решение и умозрительные результаты}
\subsection{Решение, полученное с использованием разработанного ПО}
\subsection{Выводы}
\section*{ЗАКЛЮЧЕНИЕ}
\addcontentsline{toc}{section}{ЗАКЛЮЧЕНИЕ}
\section*{СПИСОК ИСПОЛЬЗОВАННЫХ ИСТОЧНИКОВ}
\addcontentsline{toc}{section}{СПИСОК ИСПОЛЬЗОВАННЫХ ИСТОЧНИКОВ}
\section*{ПРИЛОЖЕНИЕ}
\addcontentsline{toc}{section}{ПРИЛОЖЕНИЕ}
\end{document}
