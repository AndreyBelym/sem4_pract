\documentclass[a4paper,12pt,russian]{article}
\pagestyle{empty}
\usepackage{layout}
\usepackage{amsmath}
\usepackage[left=2.5cm, right=1cm, bottom=1cm, top=1cm]{geometry}
\parindent=1.0cm
\footskip=0.5cm
%\parskip=-0.8cm
\usepackage[utf8x]{inputenc}
\usepackage[T2A]{fontenc}
\usepackage[russian,english]{babel}
\usepackage{pscyr}
\usepackage[compact,center,small]{titlesec}
\titlelabel{\thesection.  }
%\titlespacing{\section}{0.0cm}{-0.3cm}{-0.5cm}
\usepackage{listingsutf8}
\usepackage{graphicx}
\usepackage{ccaption}
\usepackage{tabularx}
\usepackage{array}
\captiondelim{ - }
\renewcommand{\baselinestretch}{1.25}
\renewcommand{\normalfont}{\fontsize{14}{20}\fontfamily{ftm}\linespread{1.25}\selectfont}
\renewcommand{\tiny}{\fontsize{12}{14}\fontfamily{ftm}\linespread{1.0}\selectfont}
\renewcommand{\maketitle}[1]{
\begin{titlepage}
\begin{center}\linespread{1}\parskip=0.0cm\normalfont
Министерство образования и науки РФ

ФГБПОУ ВПО Тульский государственный университитет

\vskip 0pt plus 0.5fil
КАФЕДРА АВТОМАТИКИ И ТЕЛЕМЕХАНИКИ

\vfill
\textbf{#1}

\vskip 2cm
Пояснительная записка 

к курсовой работе


по курсу «Программирование на ЯВУ»

\vfill
%Вариант № 4

\vfill
\begin{tabular*}{\textwidth}{ll@{\extracolsep{\fill}}c@{\extracolsep{0pt}}l}
Выполнил: & студент группы 220601&&Белым~А.А.\\ \cline{3-3}
								&&\tiny{(подпись)}& 	\\
Проверил: & к. ф.-м. н., доцент &&Сулимова В.В.\\ \cline{3-3}
								&&\tiny{(подпись)}& 	\\
\end{tabular*}
\vfill
Тула 2011
\end{center}
\end{titlepage}
}
\usepackage{float}
\newcommand{\pic}[4]{
\renewcommand{\figurename}{Рисунок}
%\suppressfloats[p]
\begin{figure}[#4]
\centering
\includegraphics{#1}
\caption{#2}\label{#3}
\end{figure}
}
\newcommand{\tabln}[1]{#1 \\ \hline}
\newcommand{\tabl}[4]{
\renewcommand{\tablename}{Таблица}
\begin{table}[#4]
\captionstyle{\raggedleft}
\caption{#1}\label{#3}
\center\normalfont
\selectlanguage{russian}
\begin{tabular}{|c|c|l|} \hline
\linespread{1.0}\selectfont
{\bfseries имя}&{\bfseries тип} &\multicolumn{1}{c|}{\bfseries предназначение}  \\ \hline
#2
\end{tabular}
\end{table}
}
\newcommand{\ftab}[1]{
 представлены в таблице \ref{#1}
}
\newcommand{\elist}[1]{
\newcounter{N}
\suppressfloats[t]
\begin{list}{\arabic{N}.}{\itemindent 1cm\addtolength{\itemindent}{\labelwidth}\listparindent 1.0cm\leftmargin 0cm\usecounter{N}}

#1
\end{list}
}
\newcommand{\prog}[2]{
\fontencoding{T2A}
\lstset{basicstyle=\tiny\fontsize{10}{12}\crfamily,keywordstyle=\bfseries,commentstyle=\itshape,identifierstyle=,stringstyle=\slshape}
\lstinputlisting[inputencoding=utf8/cp1251,language=#1]{#2}
}

\makeatletter
\renewcommand{\@oddfoot}{{\quad\hfill\quad}}
\renewcommand{\@evenfoot}{{\quad\hfill\quad}}
\renewcommand{\@dotsep}{0.5cm}
\makeatother
\newcommand{\ssec}[1]{\section{#1}\hspace*{\parindent}}
\newcommand{\ssecO}[1]{\section*{#1}\addcontentsline{toc}{section}{#1}\hspace*{\parindent}} 
 \renewcommand{\cftdot}{.}
\begin{document}
\selectlanguage{russian}
\maketitle {БЫСТРОЕ ПРЕОБРАЗОВАНИЕ ФУРЬЕ}
\ssec{Аннотация}
\normalfont
Данная   работа   является   завершающей   при   изучении   курса 
"Программирование   на   языках   высокого   уровня".   В   ней   демонстрируются 
навыки, приобретенные за все время изучения данной дисциплины.
 
В данной работе на языке программирования Delphi 7 с использованием концепций объектно-ориентированного программирования было разработано приложение, которое распознаёт ноты в файлах формата WAV, и записывает их в файл формата MIDI. Для этого используется быстрое преобразование Фурье -   алгоритм быстрого вычисления дискретного преобразования Фурье, которое позволяет разложить звуковую волну, записанную в исходном файле, на элементарные составляющие — гармонические колебания с разными частотами. Далее после анализа этих гармоник мы и получаем ноты, которые записываются в выходной файл.

Структура работы состоит из оглавления, собственно раздела о программе и численном методе , выводов по работе и списка использованной литературы.

В основном разделе производится содержательное описание задачи, её математическая формулировка и обсуждение, даллее следует описание структур данных, описание алгоритма и его разработка. После этого описываются структуры данных и функции, используемые в программе, и приводится её текст. После этого приводится умозрительные результаты при решении программой тестовой задачи, собственно решение программой задачи, и выводы по полученным программой результатам.

Итого, в работе содержится 74 страниц, 9 рисунков и 7 таблиц.
\end{document}